\section{Pregunta 01 – ¿Qué sucede al ejecutar los siguientes comandos?} 

\begin{enumerate}
\item  STARTUP OPEN\\
Inicia la instancia, monte y abra la base de datos. Esto se puede hacer en modo no restringido, permitiendo el acceso a todos los usuarios, o en modo restringido, permitiendo el acceso solo para administradores de bases de datos.
\item  STARTUP MOUNT\\
Inicia la instancia y monte la base de datos, pero déjela cerrada. Este estado permite ciertas actividades de DBA, pero no permite el acceso general a la base de datos.
\item  STARTUP NOMOUNT\\
Inicia la instancia sin montar una base de datos. Esto no permite el acceso a la base de datos y, por lo general, se haría solo para la creación de la base de datos o la recreación de archivos de control.
\item  STARTUP FORCE\\
Obliga a la instancia a iniciarse después de un problema de inicio o apagado.
\item  STARTUP RESTRICT\\
Puede iniciar una instancia y, opcionalmente, montar y abrir una base de datos, en modo restringido, de modo que la instancia solo esté disponible para el personal administrativo (no para usuarios de bases de datos generales).
\item  STARTUP RECOVER\\
Inicie la instancia y haga que la recuperación completa de los medios comience de inmediato.
\item  SHUTDOWN NORMAL\\
Para cerrar una base de datos en situaciones normales, use el comando shutdown con la cláusula normal
\item  SHUTDOWN TRANSACTIONAL\\
Cuando desee realizar un cierre planificado de una instancia mientras permite que las transacciones activas se completen primero, use el comando shutdown con la cláusula transaccional.
\item  SHUTDOWN ABORT\\
Cuando deba cerrar una base de datos abortando transacciones y conexiones de usuario, ejecute el comando shutdown con la cláusula de cancelación
\item  SHUTDOWN INMEDIATE\\
Para cerrar una base de datos inmediatamente, use el comando shutdown con la cláusula inmediata.
\end{enumerate}